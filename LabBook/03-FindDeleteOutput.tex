\section{Find, Delete and Output}
    \subsection*{Question}
        Add the following functionality to the \mintinline{cpp}{AddressBookSLL}
        class:\\

        \begin{itemize}
            \item The ability to find a person from the SLL by using their name.
                This method should return the \mintinline{cpp}{PersonNode} 
                if the \mintinline{cpp}{PersonNode} is found,
                or return \mintinline{cpp}{nullptr} if it is not found:\\
                \begin{minted}[autogobble]{cpp}
                    const PersonNode* findPerson(const string& name) const;
                \end{minted}

            \item The ability to delete a person from the SLL by using their name.
                This method should return \mintinline{cpp}{true} 
                if the \mintinline{cpp}{PersonNode} was deleted,
                or return \mintinline{cpp}{false} if it is not deleted (i.e. not found).
                \begin{minted}[autogobble]{cpp}
                    bool deletePerson(const string& name);
                \end{minted}

            \item The ability to output all of the people’s names and ages 
                in the \mintinline{cpp}{AddressBookSLL} to an \mintinline{cpp}{ostream}.
        \end{itemize}

    \subsection*{Reflection}
        \subsubsection*{Find}
            The \mintinline{cpp}{FindPerson} function works using a single for loop.
            Each iteration the current pointer traverses along the SLL,
            if the current pointer is a \mintinline{cpp}{nullptr} the for loop terminates,
            returning a \mintinline{cpp}{nullptr}.
            The current node is then checked against the searched for name and returned if a match is found.

        \subsubsection*{Delete}
            The \mintinline{cpp}{DeletePerson} function works similary to the 
            \mintinline{cpp}{FindPerson} function. However, it cannot utilize the \mintinline{cpp}{FindPerson}
            function, as in order to delete a \mintinline{cpp}{PersonNode},
            the previous node must be known to fill the gap, 
            and avoid accessing uninitialized memory.

        \subsubsection*{Output}
            The \mintinline{cpp}{ostream} output works by traversing the SLL and outputting
            each \mintinline{cpp}{PersonNode} recursively until the tail is reached.

    \subsection*{Solution}
        \begin{listing}[H]
            \inputminted[firstline=44, lastline=51]{cpp}{../Tasks/03-FindDeleteOutput/AddressBookSLL.cpp}
            \caption{Find Person}
        \end{listing}

        \begin{listing}[H]
            \inputminted[firstline=53, lastline=81]{cpp}{../Tasks/03-FindDeleteOutput/AddressBookSLL.cpp}
            \caption{Delte Person}
        \end{listing}

        \begin{listing}[H]
            \inputminted[firstline=83, lastline=89]{cpp}{../Tasks/03-FindDeleteOutput/AddressBookSLL.cpp}
            \inputminted[firstline=42]{cpp}{../Tasks/03-FindDeleteOutput/PersonNode.cpp}
            \caption{Output}
        \end{listing}